\documentclass[../Main.tex]{subfiles}
\begin{document}

\subsection{Motivation}
The recent innovations in the field of neural networks created new possibilities in the domain of video and image processing. 

We find that there still is a lack in the real-time video processing area, as well as in easy access for non-skilled users. Even the latest works we base on, provide algorithms that are not fast enough for high-resolution processing, neither for processing on average machine architectures. The deficit of availability of free and easy-to-use applications or platforms is also noticeable.


\subsection{Project objective}https://www.overleaf.com/project/5e283410b0f07700015bb003

The overall objective of our thesis is to enable fast real-time video processing using artistic style transfer by speeding up the existing convolutional neural network and ensure easy access by mobile application and lightweight server.

We aim to maintain good video quality (1024x576) while reducing the computational time by use of network pruning and TensorRT. The speedup must go hand in hand with video stability and color preservation although we do not want to depend on the sophisticated architecture. Moreover, we attempted to achieve a movie-like framerate (\textasciitilde{}24 FPS) without limiting the number of possible styles.

Access to our program should be provided for anyone interested - especially for those without programming skills necessary to launch neural network model. Therefore designing a user-friendly and cross-platform mobile application was the second goal. And since we focus on the video quality, the server should seamlessly connect application and algorithm - from any place without delays - what was the third objective.

-----
        Recent improvements of point cloud processing techniques enabled efficient and precise computation of an individual tree shape parameters, such as breast-height diameter, height, and volume. 
         I selected bark texture for the classification criteria, since they clearly represent unique characteristics of each tree and do not change their appearance under seasonable variation and aged deterioration. 
         
        However, despite all these efforts the majority of the plant species still remain without pictures or are poorly illustrated. Outside the institutional
channels, a much larger number of plant pictures are available and spread
on the web through botanist blogs, plant lovers web-pages, image hosting
websites and on-line plant retailers. 

\newpage
\subsection{Repositories}

    Each part of the project can be found on GitHub:

    \begin{itemize}
        \item{
        Neural Networks \cite{vgg} \\
        \url{https://github.com/bbbrtk/barknet-classification}
        }
        \item{ 
        Mobile Application \\
        \url{https://github.com/bbbrtk/ionic-app}
            }
        \item {
        Server \\
        \url{https://github.com/bbbrtk/flask-server}
            }
        \item {
        Thesis \\
        \url{https://github.com/bbbrtk/master-thesis}
        }
    \end{itemize}
\biblio % Needed for referencing to working when compiling individual subfiles - Do not remove
\end{document}
